\chapter{MySQL基准测试}
基准测试(benchmark)是 MySQL 新手和专家都需要掌握的一项基本技能。简单地
说,基准测试是针对系统设计的一种压力测试。通常的目标是为了掌握系统的行为。
但也有其他原因,如重现某个系统状态,或者是做新硬件的可靠性测试。本章将讨论
MySQL 和基于 MySQL的应用的基准测试的重要性、策略和工具。我们将特别讨论一下
sysbench,这是一款非常优秀的MySQL 基准测试工具。

\section{为什么需要基准测试}
为什么基准测试很重要?因为基准测试是唯一方便有效的、可以学习系统在给定的工作
负载下会发生什么的方法。基准测试可以观察系统在不同压力下的行为,评估系统的容
量,掌握哪些是重要的变化,或者观察系统如何处理不同的数据。基准测试可以在系统
实际负载之外创造一些虚构场景进行测试。基准测试可以完成以下工作,或者更多:

• 验证基于系统的一些假设,确认这些假设是否符合实际情况。

• 重现系统中的某些异常行为,以解决这些异常。

• 测试系统当前的运行情况。如果不清楚系统当前的性能,就无法确认某些优化的效
果如何。也可以利用历史的基准测试结果来分析诊断一些无法预测的问题。

• 模拟比当前系统更高的负载,以找出系统随着压力增加而可能遇到的扩展性瓶颈。

• 规划未来的业务增长。基准测试可以评估在项目未来的负载下,需要什么样的硬件,
需要多大容量的网络,以及其他相关资源。这有助于降低系统升级和重大变更的

风险。

• 测试应用适应可变环境的能力。例如,通过基准测试,可以发现系统在随机的并发
峰值下的性能表现,或者是不同配置的服务器之间的性能表现。基准测试也可以测

• 测试不同的硬件、软件和操作系统配置。比如RAID 5还是 RAID 10更适合当前的
系统?如果系统从 ATA 硬盘升级到SAN存储,对于随机写性能有什么帮助?Linux

\section{系列的内核会比2.6系列的可扩展性更好吗?升级 MySQL 的版本能改善性能吗?}
为当前的数据采用不同的存储引擎会有什么效果?所有这类问题都可以通过专门的

基准测试来获得答案。

• 证明新采购的设备是否配置正确。笔者曾经无数次地通过基准测试来对新系统进行
压测,发现了很多错误的配置,以及硬件组件的失效等问题。因此在新系统正式上

线到生产环境之前进行基准测试是一个好习惯,永远不要相信主机提供商或者硬件

供应商的所谓系统已经安装好,并且能运行多快的说法。如果可能,执行实际的基

准测试永远是一个好主意。

基准测试还可以用于其他目的,比如为应用创建单元测试套件。但本章我们只关注与性
能有关的基准测试。

基准测试的一个主要问题在于其不是真实压力的测试。基准测试施加给系统的压力相对
真实压力来说,通常比较简单。真实压力是不可预期而且变化多端的,有时候情况会过
于复杂而难以解释。所以使用真实压力测试,可能难以从结果中分析出确切的结论。

基准测试的压力和真实压力在哪些方面不同?有很多因素会影响基准测试,比如数据量、
数据和查询的分布,但最重要的一点还是基准测试通常要求尽可能快地执行完成,所以
经常给系统造成过大的压力。在很多案例中,我们都会调整给测试工具的最大压力,以
在系统可以容忍的压力阈值内尽可能快地执行测试,这对于确定系统的最大容量非常有
帮助。然而大部分压力测试工具不支持对压力进行复杂的控制。务必要记住,测试工具
自身的局限也会影响到结果的有效性。

使用基准测试进行容量规划也要掌握技巧,不能只根据测试结果做简单的推断。例如,
假设想知道使用新数据库服务器后,系统能够支撑多大的业务增长。首先对原系统进行
基准测试,然后对新系统做测试,结果发现新系统可以支持原系统40倍的 TPS(每秒
事务数),这时候就不能简单地推断说新系统一定可以支持40倍的业务增长。这是因为
在业务增长的同时,系统的流量、用户、数据以及不同数据之间的交互都在增长,它们
不可能都有40倍的支撑能力,尤其是相互之间的关系。而且当业务增长到40倍时,应
用本身的设计也可能已经随之改变。可能有更多的新特性会上线,其中某些特性可能对
数据库造成的压力远大于原有功能。而这些压力、数据、关系和特性的变化都很难模拟,
所以它们对系统的影响也很难评估。

结论就是,我们只能进行大概的测试,来确定系统大致的余量有多少。当然也可以做一
些真实压力测试(和基准测试有区别),但在构造数据集和压力的时候要特别小心,而
且这样就不再是基准测试了。基准测试要尽量简单直接,结果之间容易相互比较,成本
低且易于执行。尽管有诸多限制,基准测试还是非常有用的(只要搞清楚测试的原理,
并且了解如何分析结果所代表的意义)。

\section{基准测试的策略}
基准测试有两种主要的策略:一是针对整个系统的整体测试,另外是单独测试 MySQL。
这两种策略也被称为集成式(full-stack)以及单组件式(single-component)基准测试。
针对整个系统做集成式测试,而不是单独测试 MySQL 的原因主要有以下几点:

• 测试整个应用系统,包括Web 服务器、应用代码、网络和数据库是非常有用的,因
为用户关注的并不仅仅是 MySQL 本身的性能,而是应用整体的性能。

• MySQL 并非总是应用的瓶颈,通过整体的测试可以揭示这一点。

• 只有对应用做整体测试,才能发现各部分之间的缓存带来的影响。

• 整体应用的集成式测试更能揭示应用的真实表现,而单独组件的测试很难做到这一
点。

另外一方面,应用的整体基准测试很难建立,甚至很难正确设置。如果基准测试的设计
有问题,那么结果就无法反映真实的情况,从而基于此做的决策也就可能是错误的。

不过,有时候不需要了解整个应用的情况,而只需要关注MySQL的性能,至少在项目
初期可以这样做。基于以下情况,可以选择只测试 MySQL:

• 需要比较不同的 schema或查询的性能。

• 针对应用中某个具体问题的测试。

•

为了避免漫长的基准测试,可以通过一个短期的基准测试,做快速的“周期循环”,

来检测出某些调整后的效果。

另外,如果能够在真实的数据集上执行重复的查询,那么针对 MySQL 的基准测试也是
有用的,但是数据本身和数据集的大小都应该是真实的。如果可能,可以采用生产环境
的数据快照。

不幸的是,设置一个基于真实数据的基准测试复杂而且耗时。如果能得到一份生产数据
集的拷贝,当然很幸运,但这通常不太可能。比如要测试的是一个刚开发的新应用,它
~

只有很少的用户和数据。如果想测试该应用在规模扩张到很大以后的性能表现,就只能
通过模拟大量的数据和压力来进行。

\subsection{测试何种指标}
在开始执行甚至是在设计基准测试之前,需要先明确测试的目标。测试目标决定了选择
什么样的测试工具和技术,以获得精确而有意义的测试结果。可以将测试目标细化为一
系列的问题,比如,“这种 CPU 是否比另外一种要快?”,或 “新索引是否比当前索引性
能更好?”

有时候需要用不同的方法测试不同的指标。比如,针对延迟(latency)和吞吐量
(throughput)就需要采用不同的测试方法。

请考虑以下指标,看看如何满足测试的需求。

吞吐量

吞吐量指的是单位时间内的事务处理数。这一直是经典的数据库应用测试指标。一

些标准的基准测试被广泛地引用,如TPC-C(参考 http://www.tpc.org),而且很多数

据库厂商都努力争取在这些测试中取得好成绩。这类基准测试主要针对在线事务处

理(OLTP)的吞吐量,非常适用于多用户的交互式应用。常用的测试单位是每秒事

务数(TPS),有些也采用每分钟事务数(TPM)。

响应时间或者延迟

这个指标用于测试任务所需的整体时间。根据具体的应用,测试的时间单位可能是

微秒、毫秒、秒或者分钟。根据不同的时间单位可以计算出平均响应时间、最小

响应时间、最大响应时间和所占百分比。最大响应时间通常意义不大,因测试时

间越长,最大响应时间也可能越大。而且其结果通常不可重复,每次测试都可能得

到不同的最大响应时间。因此,通常可以使用百分比响应时间 (percentile response

time)来替代最大响应时间。例如,如果95%的响应时间都是5毫秒,则表示任务

在95%的时间段内都可以在5毫秒之内完成。

使用图表有助于理解测试结果。可以将测试结果绘制成折线图(比如平均值折线或

者95%百分比折线)或者散点图,直观地表现数据结果集的分布情况。通过这些图

可以发现长时间测试的趋势。本章后面将更详细地讨论这一点。

并发性

并发性是一个非常重要又经常被误解和误用的指标。例如,它经常被表示成多少用

户在同一时间浏览一个 web 站点,经常使用的指标是有多少个会话进。然而,HTTP

协议是无状态的,大多数用户只是简单地读取浏览器上显示的信息,这并不等同于

Web 服务器的并发性。而且,Web 服务器的并发性也不等同于数据库的并发性,而

仅仅只表示会话存储机制可以处理多少数据的能力。Web 服务器的并发性更准确的

度量指标,应该是在任意时间有多少同时发生的并发请求。

注1:特别是一些论坛软件,已经让很多管理员错误地相信同时有成千上万的用户正在同时访问网站。

在应用的不同环节都可以测量相应的并发性。Web 服务器的高并发,一般也会导

致数据库的高并发,但服务器采用的语言和工具集对此都会有影响。注意不要将

创建数据库连接和并发性搞混淆。一个设计良好的应用,同时可以打开成百上千

个 MySQL 数据库服务器连接,但可能同时只有少数连接在执行查询。所以说,一

个web 站点“同时有50000个用户”访问,却可能只有10~15个并发请求到

MySQL数据库。

换句话说,并发性基准测试需要关注的是正在工作中的并发操作,或者是同时工作

中的线程数或者连接数。当并发性增加时,需要测量吞吐量是否下降,响应时间是

否变长,如果是这样,应用可能就无法处理峰值压力。

并发性的测量完全不同于响应时间和吞吐量。它不像是一个结果,而更像是设置基

准测试的一种属性。并发性测试通常不是为了测试应用能达到的并发度,而是为了

测试应用在不同并发下的性能。当然,数据库的并发性还是需要测量的。可以通过

sysbench 指定32、64或者128个线程的测试,然后在测试期间记录MySQL 数据库

的 Threads\_running状态值。在第11 章将讨论这个指标对容量规划的影响。

扩展性

在系统的业务压力可能发生变化的情况下,测试可扩展性就非常必要了。第11章将

更进一步讨论可扩展性的话题。简单地说,可扩展性指的是,给系统增加一倍的工

作,在理想情况下就能获得两倍的结果(即吞吐量增加一倍)。或者说,给系统增加

一倍的资源(比如两倍的CPU数),就可以获得两倍的吞吐量。当然,同时性能(响

应时间)也必须在可以接受的范围内。大多数系统是无法做到如此理想的线性扩展的。

随着压力的变化,吞吐量和性能都可能越来越差。

可扩展性指标对于容量规范非常有用,它可以提供其他测试无法提供的信息,来帮

助发现应用的瓶颈。比如,如果系统是基于单个用户的响应时间测试(这是一个很

糟糕的测试策略)设计的,虽然测试的结果很好,但当并发度增加时,系统的性能

有可能变得非常糟糕。而一个基于不断增加用户连接的情况下的响应时间测试则可

以发现这个问题。

一些任务,比如从细粒度数据创建汇总表的批量工作,需要的是周期性的快速响应

时间。当然也可以测试这些任务纯粹的响应时间,但要注意考虑这些任务之间的相

互影响。批量工作可能导致相互之间有影响的查询性能变差,反之亦然。

根结底,应该测试那些对用户来说最重要的指标。因此应该尽可能地去收集一些需求,
如,什么样的响应时间是可以接受的,期待多少的并发性,等等。然后基于这些需求
设计基准测试,避免目光短浅地只关注部分指标,而忽略其他指标。

\section{基准测试方法}
在了解基本概念之后,现在可以来具体讨论一下如何设计和执行基准测试。但在讨论如
何设计好的基准测试之前,先来看一下如何避免一些常见的错误,这些错误可能导致测
试结果无用或者不精确:

• 使用真实数据的子集而不是全集。例如应用需要处理几百 GB的数据,但测试只有
1GB数据;或者只使用当前数据进行测试,却希望模拟未来业务大幅度增长后的情

况。

• 使用错误的数据分布。例如使用均匀分布的数据测试,而系统的真实数据有很多热
点区域(随机生成的测试数据通常无法模拟真实的数据分布)。

• 使用不真实的分布参数,例如假定所有用户的个人信息(profile)都会被平均地读
取 2

。

• 在多用户场景中,只做单用户的测试。

• 在单服务器上测试分布式应用。

• 与真实用户行为不匹配。例如Web 页面中的“思考时间”。真实用户在请求到一个
页面后会阅读一段时间,而不是不停顿地一个接一个点击相关链接。

• 反复执行同一个查询。真实的查询是不尽相同的,这可能会导致缓存命中率降低。
而反复执行同一个查询在某种程度上,会全部或者部分缓存结果。

• 没有检查错误。如果测试的结果无法得到合理的解释,比如一个本应该很慢的查询
突然变快了,就应该检查是否有错误产生。否则可能只是测试了MySQL 检测语法

错误的速度了。基准测试完成后,一定要检查一下错误日志,这应当是基本的要求。

• 忽略了系统预热(warm up)的过程。例如系统重启后马上进行测试。有时候需要
了解系统重启后需要多长时间才能达到正常的性能容量,要特别留意预热的时长。

反过来说,如果要想分析正常的性能,需要注意,若基准测试在重启以后马上启动,

则缓存是冷的、还没有数据,这时即使测试的压力相同,得到的结果也和缓存已经

装满数据时是不同的。

•

使用默认的服务器配置。第3章将详细地讨论服务器的优化配置。

.

测试时间太短。基准测试需要持续一定的时间。后面会继续讨论这个话题。

只有避免了上述错误,才能走上改进测试质量的漫漫长路。

如果其他条件相同,就应努力使测试过程尽可能地接近真实应用的情况。当然,有时候
和真实情况稍有些出入问题也不大。例如,实际应用服务器和数据库服务器分别部署在
不同的机器。如果采用和实际部署完全相同的配置当然更真实,但也会引入更多的变化
因素,比如加人了网络的负载和速度等。而在单一节点上运行测试相对要容易,在某些
注2:

Justin Bieber,我们爱你。这只是开个玩笑。

情况下结果也可以接受,那么就可以在单一节点上进行测试。当然,这样的选择需要根
据实际情况来分析是否合适。

\subsection{设计和规划基准测试}
规划基准测试的第一步是提出问题并明确目标。然后决定是采用标准的基准测试,还是
设计专用的测试。

如果采用标准的基准测试,应该确认选择了合适的测试方案。例如,不要使用TPC-H测
试电子商务系统。在TPC的定义中,“TPC-H 是即席查询和决策支持型应用的基准测试”,
因此不适合用来测试 OLTP 系统。

设计专用的基准测试是很复杂的,往往需要一个迭代的过程。首先需要获得生产数据集
的快照,并且该快照很容易还原,以便进行后续的测试。

然后,针对数据运行查询。可以建立一个单元测试集作为初步的测试,并运行多遍。但
是这和真实的数据库环境还是有差别的。更好的办法是选择一个有代表性的时间段,比
如高峰期的一个小时,或者一整天,记录生产系统上的所有查询。如果时间段选得比
较小,则可以选择多个时间段。这样有助于覆盖整个系统的活动状态,例如每周报表的
查询、或者非峰值时间运行的批处理作业进3

。

可以在不同级别记录查询。例如,如果是集成式(full-stack)基准测试,可以记录 Web
服务器上的HTTP 请求,也可以打开 MySQL 的查询日志(Query Log)。倘若要重演这
些查询,就要确保创建多线程来并行执行,而不是单个线程线性地执行。对日志中的每
个连接都应该创建独立的线程,而不是将所有的查询随机地分配到一些线程中。查询日
志中记录了每个查询是在哪个连接中执行的。

即使不需要创建专用的基准测试,详细地写下测试规划也是必需的。测试可能要多次反
复运行,因此需要精确地重现测试过程。而且也应该考虑到未来,执行下一轮测试时可
能已经不是同一个人了。即使还是同一个人,也有可能不会确切地记得初次运行时的情
况。测试规划应该记录测试数据、系统配置的步骤、如何测量和分析结果,以及预热的
方案等。

应该建立将参数和结果文档化的规范,每一轮测试都必须进行详细记录。文档规范可以
很简单,比如采用电子表格(spreadsheet)或者记事本形式,也可以是复杂的自定义的
数据库。需要记住的是,经常要写一些脚本来分析测试结果,因此如果能够不用打开电
子表格或者文本文件等额外操作,当然是更好的。

注3:当然,做这么多的前提是希望获得完美的基准测试结果,实际情况通常不会很顺利。

\subsection{基准测试应该运行多长时间}
基准测试应该运行足够长的时间,这一点很重要。如果需要测试系统在稳定状态时的性
能,那么当然需要在稳定状态下测试并观察。而如果系统有大量的数据和内存,要达到
稳定状态可能需要非常长的时间。大部分系统都会有一些应对突发情况的余量,能够吸
收性能尖峰,将一些工作延迟到高峰期之后执行。但当对机器加压足够长时间之后,这
些余量会被消耗尽,系统的短期尖峰也就无法维持原来的高性能。

有时候无法确认测试需要运行多长的时间才足够。如果是这样,可以让测试一直运行,
持续观察直到确认系统已经稳定。下面是一个在已知系统上执行测试的例子,图2-1显
示了系统磁盘读和写吞吐量的时序图。

2500

1250

1000

.

750

500

250

3/27

04:00

3017

05:00

3117

0 :00

3/27

13:00

327

12:00

3/17

14:40

3/27

26:09

Le axs anas:

3/7

28:00

317

20:00

Fnhhaxe inrs Opse

國 WritelOPs 圈 ReadIOPS

图2-1:扩展基准测试的1/0性能图

系统预热完成后,读1/0活动在三四个小时后曲线趋向稳定,但写1/0至少在八小时内
变化还是很大,之后有一些点的波动较大,但读和写总体来说基本稳定了进4。一个简单
的测试规则,就是等系统看起来稳定的时间至少等于系统预热的时间。本例中的测试持
续了72个小时才结束,以确保能够体现系统长期的行为。

一个常见的错误的测试方式是,只执行一系列短期的测试,比如每次60秒,并在此测
试的基础上去总结系统的性能。我们经常可以听到类似这样的话:“我尝试对新版本做
注4:顺便说一下,写1/O 的活动图展示的性能非常差。这个系统的稳定状态从性能上来说是一种灾难。
已经达到“稳定”可以说是笑话,不过这里我们的重点在于说明系统的长期行为。

了测试,但还不如旧版本快”,然而我们分析实际的测试结果后发现,测试的方式根本
不足以得出这样的结论。有时候人们也会强调说不可能有时间去测试8或者12个小时,
以验证10个不同并发性在两到三个不同版本下的性能。如果没有时间去完成准确完整
的基准测试,那么已经花费的所有时间都是一种浪费。有时候要相信别人的测试结果,
这总比做一次半拉子的测试来得到一个错误的结论要好。

\subsection{获取系统性能和状态}
在执行基准测试时,需要尽可能多地收集被测试系统的信息。最好为基准测试建立一个
目录,并且每执行一轮测试都创建单独的子目录,将测试结果、配置文件、测试指标、
脚本和其他相关说明都保存在其中。即使有些结果不是目前需要的,也应该先保存下来。
多余一些数据总比缺乏重要的数据要好,而且多余的数据以后也许会用得着。需要记录
的数据包括系统状态和性能指标,诸如CPU使用率、磁盘1/0、网络流量统计、SHOW
GLOBAL STATUS 计数器等。

下面是一个收集 MySQL 测试数据的 shell 脚本:

#!/bin/sh

INTERVAL=5

PREFIX=$INTERVAL-sec-status

RUNFILE=/home/benchmarks/running

mysql -e 'SHON GLOBAL VARIABLES'>>mysq1-variables

while test -e $RUNFILE;do

file=$(date +%F\_%I)

sleep=$(date +%s.%N | awk "{print $INTERVAL - (\$1 % $INTERVAL)}")

Sleep $sleep

ts="$(date +"TS %s.%N %F %T")"

loadavg="$(uptime)"

echo "$ts $loadavg" >>$PREFIX-${file)-status

mysql -e 'SHOW GLOBAL STATUS'>>$PREFIX-${file}-status &

echo "$ts $loadavg">>$PREFIX-${file}-innodbstatus

mysql -e 'SHOW ENGINE INNODB STATUS\G'>>$PREFIX-${file}-innodbstatus &

echo "$ts $loadavg" >>$PREFIX-${file}-processlist

mysql -e 'SHOW FULL PROCESSLIST\G'>> $PREFIX-${file}-processlist &

echo $ts

done

echo Exiting because $RUNFILE does not exist.

这个 shell 脚本很简单,但提供了一个有效的收集状态和性能数据的框架。看起来好像作
用不大,但当需要在多个服务器上执行比较复杂的测试的时候,要回答以下关于系统行
为的问题,没有这种脚本的话就会很困难了。下面是这个脚本的一些要点:

• 选代是基于固定时间间隔的,每隔5秒运行一次收集的动作,注意这里 sleep 的时间
有一个特殊的技巧。如果只是简单地在每次循环时插人一条“sleep 5”的指令,循

环的执行间隔时间一般都会稍大于5秒,那么这个脚本就没有办法通过其他脚本和

图形简单地捕获时间相关的准确数据。即使有时候循环能够恰好在5秒内完成,但

如果某些系统的时间戳是 15:32:18.218192,另外一个则是15:32:23.819437,这时候

就比较讨厌了。当然这里的5秒也可以改成其他的时间间隔,比如1、10、30或者

60秒。不过还是推荐使用5秒或者10秒的间隔来收集数据。

• 每个文件名都包含了该轮测试开始的日期和小时。如果测试要持续好几天,那么这
个文件可能会非常大,有必要的话需要手工将文件移到其他地方,但要分析全部结

果的时候要注意从最早的文件开始。如果只需要分析某个时间点的数据,则可以根

据文件名中的日期和小时迅速定位,这比在一个GB以上的大文件中去搜索要快捷

得多。

• 每次抓取数据都会先记录当前的时间戳,所以可以在文件中搜索某个时间点的数据。
也可以写一些awk或者sed 脚本来简化操作。

• 这个脚本不会处理或者过滤收集到的数据。先收集所有的原始数据,然后再基于此
做分析和过滤是一个好习惯。如果在收集的时候对数据做了预处理,而后续分析发

现一些异常的地方需要用到更多的原始数据,这时候就要“抓瞎” 了。

• 如果需要在测试完成后脚本自动退出,只需要删除/home/benchmarks/running 文件
即可。

这只是一段简单的代码,或许不能满足全部的需求,但却很好地演示了该如何捕获测
试的性能和状态数据。从代码可以看出,只捕获了 MySQL 的部分数据,如果需要,则
很容易通过修改脚本添加新的数据捕获。例如,可以通过 pt-diskstats 工具准5捕获 /proc/
diskstats的数据为后续分析磁盘1/O使用。

\subsection{获得准确的测试结果}
获得准确测试结果的最好办法,是回答一些关于基准测试的基本问题:是否选择了正确
的基准测试?是否为问题收集了相关的数据?是否采用了错误的测试标准?例如,是否
对一个1/0 密集型 (1/0-bound)的应用,采用了 CPU 密集型(CPU-bound)的测试标准
来评估性能?

接着,确认测试结果是否可重复。每次重新测试之前要确保系统的状态是一致的。如果
是非常重要的测试,甚至有必要每次测试都重启系统。一般情况下,需要测试的是经过
预热的系统,还需要确保预热的时间足够长(请参考前面关于基准测试需要运行多长时
间的内容)、是否可重复。如果预热采用的是随机查询,那么测试结果可能就是不可重复的。
如果测试的过程会修改数据或者schema,那么每次测试前,需要利用快照还原数据。在
注5:

关于 pt-dishstats 工具的更多信息,请参考第9章。

表中插入1000条记录和插入100万条记录,测试结果肯定不会相同。数据的碎片度和
在磁盘上的分布,都可能导致测试是不可重复的。一个确保物理磁盘数据的分布尽可能
一致的办法是,每次都进行快速格式化并进行磁盘分区复制。

要注意很多因素,包括外部的压力、性能分析和监控系统、详细的日志记录、周期性作
业,以及其他一些因素,都会影响到测试结果。一个典型的案例,就是测试过程中突然
有cron 定时作业启动,或者正处于一个巡查读取周期 (Patrol Read cycle),抑或 RAID
卡启动了定时的一致性检查等。要确保基准测试运行过程中所需要的资源是专用于测试
的。如果有其他额外的操作,则会消耗网络带宽,或者测试基于的是和其他服务器共享
的SAN 存储,那么得到的结果很可能是不准确的。

每次测试中,修改的参数应该尽量少。如果必须要一次修改多个参数,那么可能会丢失
一些信息。有些参数依赖其他参数,这些参数可能无法单独修改。有时候甚至都没有意
识到这些依赖,这给测试带来了复杂性进6。

一般情况下,都是通过迭代逐步地修改基准测试的参数,而不是每次运行时都做大量
的修改。举个例子,如果要通过调整参数来创造一个特定行为,可以通过使用分治法
(divide-and-conquer,每次运行时将参数对分减半)来找到正确的值。

很多基准测试都是用来做预测系统迁移后的性能的,比如从 Oracle 迁移到MySQL。这
种测试通常比较麻烦,因 MySQL 执行的查询类型与 Oracle完全不同。如果想知道在
Oracle运行得很好的应用迁移到 MySQL 以后性能如何,通常需要重新设计 MySQL的
schema 和查询(在某些情况下,比如,建立一个跨平台的应用时,可能想知道同一条查
询是如何在两个平台运行的,不过这种情况并不多见)。

另外,基于MySQL 的默认配置的测试没有什么意义,因为默认配置是基于消耗很少内
存的极小应用的。有时候可以看到一些MySQL 和其他商业数据库产品的对比测试,结
果很让人尴尬,可能就是 MySQL 采用了默认配置的缘故。让人无语的是,这样明显有
误的测试结果还容易变成头条新闻。

固态存储(SSD或者PCI-E卡)给基准测试带来了很大的挑战,第9章将进一步讨论。

最后,如果测试中出现异常结果,不要轻易当作坏数据点而丢弃。应该认真研究并找到
产生这种结果的原因。测试可能会得到有价值的结果,或者一个严重的错误,抑或基准
测试的设计缺陷。如果对测试结果不了解,就不要轻易公布。有一些案例表明,异常的
测试结果往往都是由于很小的错误导致的,最后搞得测试无功而返进”。

注6:

有时,这并不是问题。例如,如果正在考虑从基于 SPARC 的 Solaris 系统迁移到基于 x86的GNU/

Linux 系统,就没有必要测试基于 x86的 Solaris 作为中间过程。

注7:

本书的任何一位作者都还没发生过这样的事情,仅供参考。

\subsection{运行基准测试并分析结果}
一旦准备就绪,就可以着手基准测试,收集和分析数据了。

通常来说,自动化基准测试是个好主意。这样做可以获得更精确的测试结果。因为自动
化的过程可以防止测试人员偶尔遗漏某些步骤,或者误操作。另外也有助于归档整个测
试过程。

自动化的方式有很多,可以是一个 Makefile 文件或者一组脚本。脚本语言可以根据需要
选择:shell、PHP、Perl 等都可以。要尽可能地使所有测试过程都自动化,包括装载数据、
系统预热、执行测试、记录结果等。

一旦设置了正确的自动化操作,基准测试将成为一步式操作。如果只是针对某些应

用做一次性的快速验证测试,可能就没必要做自动化。但只要未来可能会引用到测

试结果,建议都尽量地自动化。否则到时候可能就搞不清楚是如何获得这个结果的,

也不记得采用了什么参数,这样就很难再通过测试重现结果了。

基准测试通常需要运行多次。具体需要运行多少次要看对结果的记分方式,以及测试的
重要程度。要提高测试的准确度,就需要多运行几次。一般在测试的实践中,可以取最
好的结果值,或者所有结果的平均值,抑或从五个测试结果里取最好三个值的平均值。
可以根据需要更进一步精确化测试结果。还可以对结果使用统计方法,确定置信区间
(confidence interval)等。不过通常来说,不会用到这种程度的确定性结果进B。只要测试
的结果能满足目前的需求,简单地运行几轮测试,看看结果的变化就可以了。如果结果
变化很大,可以再多运行几次,或者运行更长的时间,这样都可以获得更确定的结果。

获得测试结果后,还需要对结果进行分析,也就是说,要把“数字”变成“知识”

“。墩

终的目的是回答在设计测试时的问题。理想情况下,可以获得诸如“升级到4核CPU可
以在保持响应时间不变的情况下获得超过50%的吞吐量增长”或者“增加索引可以使查
询更快”的结论。如果需要更加科学化,建议在测试前读读 null hypothesis 一书,但大
部分情况下不会要求做这么严格的基准测试。

如何从数据中抽象出有意义的结果,依赖于如何收集数据。通常需要写一些脚本来分析
数据,这不仅能减轻分析的工作量,而且和自动化基准测试一样可以重复运行,并易于
文档化。下面是一个非常简单的 shell脚本,演示了如何从前面的数据采集脚本采集到的
数据中抽取时间维度信息。脚本的输人参数是采集到的数据文件的名字。

注8:

如果真的需要科学可靠的结果,应该去读读关于如何设计和执行可控测试的书籍,这个已经超出

了本书讨论的范畴。

#!/bin/sh

# This script converts SHOW CLOBAL STATUS into a tabulated format, one line

# per sample in the input, with the metrics divided by the time elapsed

# between samples.

awk

BEGIN{

printf "#ts date time load QPs";

fmt = " %.2f";

/^TS/ {# The timestamp lines begin with TS.

ts

substr($2,1, index($2,

".")- 1);

load

= NF -2;

diff

: ts -prev\_ts;

prev\_ts

= tS;

printf "In%s %s %s %s"

,tS,$3,$4, substr($load, 1, length($load)-1);

}

/Queries/{

printf fmt, ($2-Queries)/diff;

Queries=$2

"$@"

假设该脚本名为 analyze,当前面的脚本生成状态文件以后,就可以运行该脚本,可能会
得到如下的结果:

[baron@g inger ~]$•/analyze 5-sec-status-2011-03-20

#s date time load QPS

1300642150 2011-03-20 17:29:10 0.00 0.62

1300642155 2011-03-20 17:29:15 0.00 1311.60

1300642160 2011-03-20 17:29:20 0.00 1770.60

1300642165 2011-03-20 17:29:25 0.00 1756.60

1300642170 2011-03-20 17:29:30 0.00 1752.40

1300642175 2011-03-20 17:29:35 0.00 1735.00

1300642180 2011-03-20 17:29:40 0.00 1713.00

1300642185 2011-03-20 17:29:45 0.00 1788.00

1300642190 2011-03-20 17:29:50 0.00 1596.40

第一行是列的名字,第二行的数据应该忽略,因为这是测试实际启动前的数据。接下来
的行包含 Unix时间戳、日期、时间(注意时间数据是每5秒更新一次,前面脚本说明
时曾提过)、系统负载、数据库的QPS(每秒查询次数)五列,这应该是用于分析系统
性能的最少数据需求了。接下来将演示如何根据这些数据快速地绘成图形,并分析基准
测试过程中发生了什么。

\subsection{绘图的重要性}
如果你想要统治世界,就必须不断地利用“阴谋”注’。而最简单有效的图形,就是将性能
指标按照时间顺序绘制。通过图形可以立刻发现一些问题,而这些问题在原始数据中却
注9:

英语中 plot既有“阴谋”的意思,也有“绘图”的意思,所以这里是一句双关语。——译者注

很难被注意到。或许你会坚持看测试工具打印出来的平均值或其他汇总过的信息,但平
均值有时候是没有用的,它会掩盖掉一些真实情况。幸运的是,前面写的脚本的输出都
可以定制作为 gnuplot 或者R绘图的数据来源。假设使用gnuplot,假设输出的数据文件
名是 QPS-per-5-seconds:

gnuplot>plot "OPS-per-5-seconds" using 5 w lines title “QPs"

该gnuplot 命令将文件的第五列qps数据绘成图形,图的标题是QPS。图2-2是绘制出
来的结果图。

1800

1600

1400

1200

1000

800

600 ~

400

200

0

QPS

0

2

3

4

5

6

8

图2-2:基准测试的QPS图形

下面我们讨论一个可以更加体现图形价值的例子。假设 MySQL 数据正在遭受“疯狂刷
新(furious flushing)”的问题,在刷新落后于检查点时会阻塞所有的活动,从而导致吞
吐量严重下跌。95%的响应时间和平均响应时间指标都无法发现这个问题,也就是说这
两个指标掩盖了问题。但图形会显示出这个周期性的问题,请参考图2-3。

图2-3显示的是每分钟新订单的交易量 (NOTPM,new-order transactions per minute)。
从曲线可以看到明显的周期性下降,但如果从平均值(点状虚线)来看波动很小。一开
始的低谷是由于系统的缓存是空的,而后面其他的下跌则是由于系统刷新脏块到磁盘导
致。如果没有图形,要发现这个趋势会比较困难。

12000

10000

8000

6000

4000

2000

0

10 11 12 13 14 15 16 17 18 19 20 21 22 23 24 25 26 27 28 29 30

时间,分钟

图2-3:一个30分钟的dbt2测试的结果

这种性能尖刺在压力大的系统比较常见,需要调查原因。在这个案例中,是由于使用了
旧版本的InnoDB引擊,脏块的刷新算法性能很差。但这个结论不能是想当然的,需要
认真地分析详细的性能统计。在性能下跌时,SHOW ENGINE INNODB STATUS的输出是什么?

SHOW FULL PROCESSLIST 的输出是什么?应该可以发现 InnoDB 在持续地刷新脏块,并
且阻塞了很多状态是 “waiting on query cache lock”的线程,或者其他类似的现象。在
执行基准测试的时候要尽可能地收集更多的细节数据,然后将数据绘制成图形,这样可
以帮助快速地发现问题。

\section{基准测试工具}
没有必要开发自己的基准测试系统,除非现有的工具确实无法满足需求。下面的章节会
介绍一些可用的工具。

\subsection{集成式测试工具}
回忆一下前文提供的两种测试类型:集成式测试和单组件式测试。毫不奇怪,有些工具
是针对整个应用进行测试,也有些工具是针对 MySQL 或者其他组件单独进行测试的。
集成式测试,通常是获得整个应用概况的最佳手段。已有的集成式测试工具如下所示。

ab

ab是一个 Apache HTTP服务器基准测试工具。它可以测试 HTTP服务器每秒最多

可以处理多少请求。如果测试的是Web 应用服务,这个结果可以转换成整个应用

每秒可以满足多少请求。这是个非常简单的工具,用途也有限,只能针对单个 URL

进行尽可能快的压力测试。关于 ab 的更多信息可以参考 http://httpd.apache.org/

docs/2.0/programs/ab.html。

http\_load

这个工具概念上和ab类似,也被设计为对 Web 服务器进行测试,但比 ab 要更加灵活。

可以通过一个输人文件提供多个 URL,http\_load 在这些 URL 中随机选择进行测试。

也可以定制http\_load,使其按照时间比率进行测试,而不仅仅是测试最大请求处理

能力。更多信息请参考 http://www.acme.com/softvare/http-load/。

JMeter

JMeter 是一个Java 应用程序,可以加载其他应用并测试其性能。它虽然是设计用来

测试 web 应用的,但也可以用于测试其他诸如FTP服务器,或者通过JDBC 进行数

据库查询测试。

JMeter 比ab 和http\_load 都要复杂得多。例如,它可以通过控制预热时间等参数,

更加灵活地模拟真实用户的访问。JMeter 拥有绘图接口(带有内置的图形化处理的

功能),还可以对测试进行记录,然后离线重演测试结果。更多信息请参考 http://

jakarta.apache.org/imeter/。

\subsection{单组件式测试工具}
有一些有用的工具可以测试 MySQL 和基于 MySQL 的系统的性能。2.5节将演示如何利
用这些工具进行测试。

mysglslap

mysqlslap (http://dev.mysgl.com/doc/refman/5.1/en/mysqLslap.htm!)可以模拟服务器

的负载,并输出计时信息。它包含在MySQL 5.1 的发行包中,应该在 MySQL 4.1

或者更新的版本中都可以使用。测试时可以执行并发连接数,并指定SQL 语句(可

以在命令行上执行,也可以把SQL 语句写人到参数文件中)。如果没有指定SQL语句,

mysqislap 会自动生成查询 schema的 SELECT 语句。

MySQL Benchmark Suite (sgl-bench)

在MySQL 的发行包中也提供了一款自己的基准测试套件,可以用于在不同数据库

服务器上进行比较测试。它是单线程的,主要用于测试服务器执行查询的速度。结

果会显示哪种类型的操作在服务器上执行得更快。

这个测试套件的主要好处是包含了大量预定义的测试,容易使用,所以可以很轻

松地用于比较不同存储引擎或者不同配置的性能测试。其也可以用于高层次测试,

比较两个服务器的总体性能。当然也可以只执行预定义测试的子集(例如只测试

UPDATE的性能)。这些测试大部分是CPU 密集型的,但也有些短时间的测试需要大
量的磁盘1/0操作。

这个套件的最大缺点主要有:它是单用户模式的,测试的数据集很小且用户无法使
用指定的数据,并且同一个测试多次运行的结果可能会相差很大。因为是单线程且串
行执行的,所以无法测试多CPU的能力,只能用于比较单CPU服务器的性能差别。

使用这个套件测试数据库服务器还需要 Perl 和 BDB 的支持,相关文档请参考 http://
dev.mysgl.com/doc/en/mysgl-benchmarks.html/。

?r Smack

Super Smack (http://vegan.net/tony/supersmack/)是一款用于 MySQL 和 PostgreSQL
的基准测试工具,可以提供压力测试和负载生成。这是一个复杂而强大的工具,可
以模拟多用户访问,可以加载测试数据到数据库,并支持使用随机数据填充测试表。
测试定义在“

smack”文件中,smack 文件使用一种简单的语法定义测试的客户端、

表、查询等测试要素。

abase Test Suite

Database Test Suite 是由开源软件开发实验室(OSDL, Open Source Development
Labs)设计的,发布在 SourceForge 网站(http://sourceforge.net/projects/osdldbt/)
上,这是一款类似某些工业标准测试的测试工具集,例如由事务处理性能委员会
(TPC,Transaction Processing Performance Council) 制定的各种标准。特别值得一
提的是,其中的dbt2 就是一款免费的TPC-C OLTP 测试工具(未认证)。之前本书
作者经常使用该工具,不过现在已经使用自己研发的专用于 MySQL 的测试工具替
代了。

:ona's TPCC-MySQL Tool

我们开发了一个类似TPC-C的基准测试工具集,其中有部分是专门为 MySQL 测
试开发的。在评估大压力下MySQL 的一些行为时,我们经常会利用这个工具进行
测试(简单的测试,一般会采用sysbench替代)。该工具的源代码可以在 https://
launchpad.net/perconatools 下载,在源码库中有一个简单的文档说明。

ench

sysbench(https://launchpad.net/sysbench)是一款多线程系统压测工具。它可以根
据影响数据库服务器性能的各种因素来评估系统的性能。例如,可以用来测试文件
1/O、操作系统调度器、内存分配和传输速度、POSIX线程,以及数据库服务器等。
sysbench 支持 Lua 脚本语言 (http://www.lua.org),Lua对于各种测试场景的设置可
以非常灵活。sysbench 是我们非常喜欢的一种全能测试工具,支持 MySQL、操作系
统和硬件的硬件测试。

MySQL 的 BENCHMARK()函数

MySQL 有一个内置的 BENCHMARK()函数,可以测试某些特定操作的执行速度。参

数可以是需要执行的次数和表达式。表达式可以是任何的标量表达式,比如返回值

是标量的子查询或者函数。该函数可以很方便地测试某些特定操作的性能,比如通

过测试可以发现,MD5()函数比 SHA1()函数要快:

mysq1> SET @input := 'hello world';

mysqL> SELECT BENCHMARK(1000000, MD5(@input));

BENCHMARK (1000000, MD5(@input))|

01

1 IOw in set (2.78 sec)

mysq1> SELECT BENCHMARK(1000000, SHA1(@input));

BENCHMARK(1000000, SHA1(@input))|

-+

一一千

1 row in set (3.50 sec)

执行后的返回值永远是0,但可以通过客户端返回的时间来判断执行的时间。在这

个例子中可以看到MD5()执行比SHA1()要快。使用 BENCHMARK()函数来测试性能,

需要清楚地知道其原理,否则容易误用。这个函数只是简单地返回服务器执行表达

式的时间,而不会涉及分析和优化的开销。而且表达式必须像这个例子一样包含用

户定义的变量,否则多次执行同样的表达式会因为系统缓存命中而影响结果注10

虽然 BENCHMARK()函数用起来很方便,但不合适用来做真正的基准测试,因为很难

理解真正要测试的是什么,而且测试的只是整个执行周期中的一部分环节。

\section{基准测试案例}
本节将演示一些利用上面提到的基准测试工具进行测试的真实案例。这些案例未必涵盖
所有测试工具,但应该可以帮助读者针对自己的测试需要来做出判断和选择,并作为人
门的开端。

注10:本书作者之一碰到了这个问题,因为发现循环执行1000次表达式和只执行一次表达式的时间居然
差不多,这只能说明缓存命中了。实际上,当碰到此类情况时,第一反应就应当是缓存命中或者

出错了。

\subsection{http\_load}
下面通过一个简单的例子来演示如何使用 http\_load。首先创建一个 urls.txt 文件,输人
如下的 URL:

http://www.mysqlperformanceb10g.com/

http://www.mysqlperformanceblog.com/page/2/

http://www.mysqlperformanceblog.com/mysq1-patches/

http://www.mysqlperformanceb1og.com/mysql-performance-presentations/

http://www.mysqlperformanceb1og.com/2006/09/06/s1ow-query-10g-analyzes-tools/

http\_load 最简单的用法,就是循环请求给定的URL列表。测试程序将以最快的速度请
求这些 URL:

$http-Load -parallel 1 -seconds 10 urLSotxt

19 fetches,1 max parallel, 837929 bytes, in 10.0003 seconds

44101.5 mean bytes/connection

1.89995 fetches/seC,83790.7 bytes/sec

msecs/connect: 41.6647 mean,56.156 max,38.21 min

msecs/first-response: 320.207 mean,508.958 max,179.308 min

HTTP response codes:

code 200 - 19

测试的结果很容易理解,只是简单地输出了请求的统计信息。下面是另外一个稍微复杂
的测试,还是尽可能快地循环请求给定的URL 列表,不过模拟同时有五个并发用户在
进行请求:

$ http\_load -parallel 5 -seconds 10 urlsotxt

94 fetches, 5 max parallel,4.75565e+06 bytes, in 10.0005 seconds

50592 mean bytes/connection

9.39953 fetches/sec,475541 bytes/sec

msecs/connect: 65.1983 mean,169.991 max,38.189 min

msecs/first-response:245.014 mean,993.059 max,99.646 min

HTTP response codes:

code 200-94

另外,除了测试最快的速度,也可以根据预估的访问请求率(比如每秒5次)来做压大
模拟测试。

$ http\_load -rate 5 -seconds 10 urls.txt

52105 mean bytes/connection

4.8 fetches/seC,250104 bytes/sec

msecs/connect: 42.5931 mean,60.462 max,38.117 min

msecs/first-response: 246.811 mean,546.203 max,108.363 min

HTTP response codes:

code 200 -48

最后,还可以模拟更大的负载,可以将访问请求率提高到每秒20次请求。请注意,连
接和请求响应时间都会随着负载的提高而增加。

$ http\_load -rate 20 -seconds 10 urls.txt

111 fetches,89 max parallel, 5.91142e+06 bytes, in 10.0001 seconds

53256.1 mean bytes/connection

11.0998 fetches/seC,591134 bytes/sec

msecs/connect:100.384 mean,211.885 max,38.214 min

msecs/first-response: 2163.51 mean, 7862.77 max, 933.708 min

HTTP response codes:

code 200-- 111

\subsection{MySQL 基准测试套件}
MySQL 基准测试套件(MySQL Benchmark Suite)由一组基于Per!l 开发的基准测试工具
组成。在 MySQL 安装目录下的sql-bench 子目录中包含了该工具。比如在 Debian GNU/
Linux 系统上,默认的路径是 /uusr/share/mysql/sgl-bench。

在用这个工具集测试前,应该读一下 README 文件,了解使用方法和命令行参数说明。
如果要运行全部测试,可以使用如下的命令:

$ cd /usr/share/mysq1/sq1-bench/

sql-bench$./run-al1-tests --server=mys9l --user=root --log --fast

Test finished. You can find the result in:

output/RUN-mysql.

\_fast-Linux\_2.4.18.

1686.

\_smp\_i686

运行全部测试需要比较长的时间,有可能会超过一个小时,其具体长短依赖于测试的硬
件环境和配置。如果指定了--log命令行,则可以监控到测试的进度。测试的结果都保
存在output子目录中,每项测试的结果文件中都会包含一系列的操作计时信息。下面是
一个具体的例子,为方便印刷,部分格式做了修改。

5q1-bench$ tail -5 output/select-my$q\_fast-Linux\_2.4.18.686\_smp\_i686

Time for count

\_distinct

\_810up.

\_key (1000:6000):

34 wallclock secs (0.20 usr 0.08 sys + 0.00 cusr

0.00 csys = 0.28 CPU)

Time for count

-distinct\_group\_on\_key\_parts (1000:100000):

34 wallclock secs (0.57 uSI

0.27 SyS +0.00 CUSI 0.00 CSYS = 0.84 CPU)

Time tor count\_distinct\_group (1000:100000):

34

wallclock secs(0.59 usr

0.20 Sys + 0.00 CuSr 0.00 cSys = 0.79 CPU)

Time for count

distinct,

:-bi8(100:1000000):

8

wallclock secs (4.22 usr 2.20 sys + 0.00 cusr 0.00 csys = 6.42 CPU)

Total time:

868 wallclock secs (33.24 usr 9.55 sys + 0.00 CUSr 0.00 CSys = 42.79 CPU)

如上所示,count\_distinct.

:\_group\_on\_key(1000:6000)测试花费了34秒(wallclock secs),

这是客户端运行测试花费的总时间;其他值(包括usr,sy,cursr,csys)则占了测
试的0.28秒的开销,这是运行客户端测试代码所花费的时间,而不是等待 MySQL服务
器响应的时间。而测试者真正需要关心的测试结果,是除去客户端控制的部分,即实际
运行时间应该是 33.72秒。

除了运行全部测试集外,也可以选择单独执行其中的部分测试项。例如可以选择只执行
insert 测试,这会比运行全部测试集所得到的汇总信息给出更多的详细信息:

sql-bench$•/test-insert

Testing server "MySQL 4.0.13 1og' at 2003-05-18 11:02:39

Testing the speed of inserting data into 1 table and do some selects on it.

The tests are done with a table that has 100000 rows.

Generating random keys

Creating tables

Inserting 100000 rows in order

Inserting 100000 rows in reverse order

Inserting 100000 rows in random order

Time for insert (300000):

42 wallclock secs (7.91 usr 5.03 sys + 0.00 Cusr 0.00 CSys = 12.94 CPU)

Testing insert of duplicates

Time for insert\_duplicates (100000):

16 wallclock secs ( 2.28 usr 1.89 sys + 0.00 CUSI 0.00 CSyS = 4.17 CPU)

\subsection{sysbench}
Sysbench 可以执行多种类型的基准测试,它不仅设计用来测试数据库的性能,也可以
测试运行数据库的服务器的性能。实际上,Peter 和Vadim 最初设计这个工具是用来执
行 MySQL 性能测试的(尽管并不能完成所有的MySQL 基准测试)。下面先演示一些非
MySQL 的测试场景,来测试各个子系统的性能,这些测试可以用来评估系统的整体性
能瓶颈。后面再演示如何测试数据库的性能。

强烈建议大家都能熟悉 sysbench测试,在MySQL 用户的工具包中,这应该是最有用
的工具之一。尽管有其他很多测试工具可以替代sysbench的某些功能,但那些工具有
时候并不可靠,获得的结果也不一定和 MySQL 性能相关。例如,1/0性能测试可以用
iozone、bonniet+ 等一系列工具,但需要注意设计场景,以便可以模拟InnoDB 的磁盘
1/0模式。而sysbench 的1/O 测试则和 InnoDB 的1/O模式非常类似,所以 fileio选项是非
常好用的。

sysbench 的CPU 基准测试

最典型的子系统测试就是CPU基准测试。该测试使用64位整数,测试计算素数直到某
个最大值所需要的时间。下面的例子将比较两台不同的GNU/Linux服务器上的测试结果。
第一台机器的 CPU 配置如下:

[server1 ~]$ cat /proc/cpuinfo

•••

model name

:AMD Opteron(tm)Processor 246

stepping

:1

cpu MHz

:1992.857

cache size

:1024 KB

在这台服务器上运行如下的测试:

[server1 ~]$ sysbench --test=cpu --cpu-max-prime=20000 run

sysbench v0.4.8:

multithreaded system evaluation benchmark

Test execution summary:

total time:

第二台服务器配置了不同的CPU:

121.74045

[server2 ~]$ cat /proc/cpuinfo

model name

: Intel(R) Xeon(R) CPU

stepping

:6

cpu MHz

:1995.005

5130 @ 2.00GHz

测试结果如下:

[server1 ~]$ sysbench --test=cpu --Cpu-max-prime=20000 run

sysbench v0.4.8: multithreaded system evaluation benchmark

Test execution summary:

total time:

61.8596s

测试的结果简单打印出了计算出素数的时间,很容易进行比较。在上面的测试中,第二
台服务器的测试结果显示比第一台快两倍。

sysbench 的文件I/O 基准测试

文件1/O(fileio)基准测试可以测试系统在不同1/O 负载下的性能。这对于比较不同的
硬盘驱动器、不同的RAID卡、不同的RAID 模式,都很有帮助。可以根据测试结果来
调整I/O子系统。文件I/O 基准测试模拟了很多InnoDB 的1/O 特性。

测试的第一步是准备(prepare)阶段,生成测试用到的数据文件,生成的数据文件至少
要比内存大。如果文件中的数据能完全放入内存中,则操作系统缓存大部分的数据,与
致测试结果无法体现 1/O密集型的工作负载。首先通过下面的命令创建一个数据集:

$ sysbench --test=fileio --file-total-size=150G prepare

这个命令会在当前工作目录下创建测试文件,后续的运行(run)阶段将通过读写这些文
件进行测试。第二步就是运行(run)阶段,针对不同的I/O 类型有不同的测试选项:

seqwr

顺序写人。

seqrewr

顺序重写。

eqrd

ndrd

ndwr

dnrw

顺序读取。

随机读取。

随机写人。

混合随机读/写。

下面的命令运行文件1/0混合随机读/ 写基准测试:
$ sysbench --test=fileio --file-total-size=150G -

--init-rng=on --max-time=300 --max-requests=0 run

吉果如下:

sysbench v0.4.8: multithreaded system evaluation

Running the test with following options:

Number of threads: 1

Initializing random number generator from timer.

Extra file open fLags: 0

128 files, 1.17190b each

150Gb total file size

Block size 16Kb

Number of random requests for randon IO: 10000

Read/write ratio for combined random IO test : 1.5

Periodic FSYNC enabled, calling fsync() each 100

Calling fsync( at the end of test, Enabled.

Using synchronous I/0 mode

Doing random r/w test

Threads started!

Time 1imit exceeded, exiting.•

Done.

Operations performed: 40260 Read, 26840 Write, 8

Read 629.06Mb written 419.38Mb

Total transferre

223.67 Requests/sec executed

Test execution summary:

total time:

300.0004

total number of events:

67100

total time taken by event execution: 254.4601

per-request statistics:

min:

avg:

0.0000s

0.0038s

max:

approx. 95 percentile:

0.5628s

0.0099s

Threads fairness:

events (avg/stddev):

execution time (avg/stddev):

67100.0000/0.00

254.4601/0.00

输出结果中包含了大量的信息。和1/0子系统密切相关的包括每秒请求数和总吞!
在上述例子中,每秒请求数是 223.67 Requests/sec,吞吐量是3.4948MB/sec。另外
间信息也非常有用,尤其是大约95%的时间分布。这些数据对于评估磁盘性能十分7
测试完成后,运行清除(cleanup)操作删除第一步生成的测试文件:

$ sysbench --test=fileio --file-total-size=150G cleanup

sysbench 的OLTP 基准测试

OLTP 基准测试模拟了一个简单的事务处理系统的工作负载。下面的例子使用的是
超过百万行记录的表,第一步是先生成这张表:

$ sysbench

--test=oltp --oltp-table-size=1000000 --mysq1-db=test/

--mysqL-user=root prepare

sysbench vo.4.8: multithreaded system evaluation benchmark

No DB drivers specified, using mysql

Creating table 'sbtest'

Creating 1000000 records in table 'sbtest'...

生成测试数据只需要上面这条简单的命令即可。接下来可以运行测试,这个例子采
8个并发线程,只读模式,测试时长60秒:

$ sysbench --test=oltp --OLtp-table-size=1000000 --mysql-db=test --mysql-user=root/

--max-time=60 --oltp-read-on\_y=on --max-requests=0 --num-threads=8 run

sysbench vo.4.8: multithreaded system evaluation benchmark

No DB drivers specified,using mysql

WARNING:Preparing of "BEGIN" is unsupported, using emulation

(last message repeated 7 times)

Running the test with following options:

Number of threads:8

Doing OLTP test.

Running mixed OLTP test

Doing read-only test

Using Special distribution (12 iterations,

1 pct of values are returned in 75 pct

cases)

Using “BEGIN" for starting transactions

Using auto\_inc on the id column

Threads started!

Time limitexceeded,exiting.•

(last message repeated 7 times)

Done.

OLTP test statistics:

queries performed:

read:

write:

other:

total:

transactions:

deadlocks:

read/write requests:

.other operations:

Test execution summary:

total time:

60.21145

total number of events:

12829

total time taken by event execution: 480.2086

per-request statistics:

min:

avg:

max:

approx. 95 percentile:

0.0030s

0.0374s

1.91065

0.11635

Threads fairness:

events (avg/stddev):

execution time (avg/stddev):

1603.6250/70.66

60.0261/0.06

179606

0

25658

205264

12829(213.07 per sec.)

。

(0.00 per sec.)

179606(2982.92 per sec.)

25658 (426.13 per sec.)

如上所示,结果中包含了相当多的信息。其中最有价值的信息如下:

•

总的事务数。

•

每秒事务数。

•

时间统计信息(最小、平均、最大响应时间,以及95% 百分比响应

•

线程公平性统计信息(thread-fairness),用于表示模拟负载的公平性

这个例子使用的是 sysbench 的第4版,在 SourceForge.net 可以下载到这
的可执行文件。也可以从Launchpad 下载最新的第5版的源代码自行编译
有用的事情),这样就可以利用很多新版本的特性,包括可以基于多个
进行测试,可以每隔一定的间隔比如10秒打印出吞吐量和响应的结果
理解系统的行为非常重要。

sysbench 的其他特性

sysbench还有一些其他的基准测试,但和数据库性能没有直接关系。

内存(memory)

测试内存的连续读写性能。

线程(thread)

测试线程调度器的性能。对于高负载情况下测试线程调度器的行为非常有用。

互斥锁(mutex)

测试互斥锁(mutex)的性能,方式是模拟所有线程在同一时刻并发运行,并都短暂

请求互斥锁(互斥锁 mutex 是一种数据结构,用来对某些资源进行排他性访问控制,

防止因并发访问导致问题)。

顺序写(seqwr)

测试顺序写的性能。这对于测试系统的实际性能瓶颈很重要。可以用来测试 RAID

控制器的高速缓存的性能状况,如果测试结果异常则需要引起重视。例如,如果

RAID 控制器写缓存没有电池保护,而磁盘的压力达到了3000次请求/秒,就是一

个问题,数据可能是不安全的。

另外,除了指定测试模式参数(--test)外,sysbench还有其他很多参数,比如--num
-threads、--max-requests 和--max-time 参数,更多信息请查阅相关文档。

\subsection{数据库测试套件中的 dbt2 TPC-C 测试}
数据库测试套件(Database Test Suite)中的dbt2是一款免费的TPC-C测试工具。
TPC-C是TPC组织发布的一个测试规范,用于模拟测试复杂的在线事务处理系统
(OLTP)。它的测试结果包括每分钟事务数(tpmC),以及每事务的成本(Price/tpmC)。
这种测试的结果非常依赖硬件环境,所以公开发布的TPC-C测试结果都会包含具体的系
统硬件配置信息。

dbt2并不是真正的TPC-C测试,它没有得到TPC组织的认证,它的结果不能直接

跟TPC-C的结果做对比。而且本书作者开发了一款比dbt2 更好的测试工具,详细

情况见2.5.5节。

下面看一个设置和运行dbt2基准测试的例子。这里使用的是 dbt2 0.37版本,这个版本
能够支持 MySQL 的最新版本(还有更新的版本,但包含了一些MySQL 不能提供完全
支持的修正)。下面是测试步骤。

1.准备测试数据。

下面的命令会在指定的目录创建用于10个仓库的数据。每个仓库使用大约700MB

磁盘空间,测试所需要的总的磁盘空间和仓库的数量成正比。因此,可以通过-W 参

数来调整仓库的个数以生成合适大小的数据集。

# srC/datagen -w 10 -d /mnt/data/dbt2-w10

warehouses = 10

districts = 10

Customers = 3000

items = 100000

orders = 3000

stock = 100000

new\_orders= 900

Output directory of data files: /mnt/data/dbt2-W10

Generating data files for 10 warehouse(s)..

Generating item table data...

Finished item table data...

Generating warehouse table data...

Finished warehouse table data.•

Generating stock table data..

加载数据到 MySQL 数据库。

下面的命令创建一个名为 dbt2w10的数据库,并且将上一步生成的测试
数据库中(-d参数指定数据库, f参数指定测试数据所在的目录)。

# scripts/mysq1/mysql\_load\_db.sh -d dbt2w10 -f /mnt/data/dbt2-w10/

-s /var/1ib/mysql/mysql.sock

运行测试。

最后一步是运行 scripts 脚本目录中的如下命令执行测试:

# run\_mysql.sh -C 10 -W 10 -t 300 -n dbt2w10/

-u root -o /var/1ib/mysq1/mysql.sock-e

**********************************************米***水水

*

*

DBT2 test for MySQL started

*

*

*

Results can be found in output/9 directory

*

************************************************************************

*

*

*

Test consists of 4 stages:

*

*

*

*

1. Start of client to create pool of databases connections

*

*

2. Start of driver to emulate terminals and transactions generation *

* 3. Test

*

* 4. Processing of results

*

*

*

*****************************************水*水*

:****************冰*

DATABASE NAME:

DATABASE USER:

DATABASE SOCKET:

DATABASE CONNECTIONS:

TERMINAL THREADS:

SCALE FACTOR (WARHOUSES):

TERMINALS PER WAREHOUSE:

DURATION OF TEST(in sec):

SLEEPY in(msec)

dbt2w10

root

/var/1ib/mysq1/mysql.sock

10

100

10

10

300

300

ZERO DELAYS MODE:

1

Stage 1. Starting up client...

Delay for each thread - 300 msec. Will sleep for 4 sec to start 10 database

connections

CLIENT\_PID = 12962

Stage 2. Starting up driver..

Delay for each thread - 300 msec. Wil1 sleep for 34 sec to start 100 terminal

A11 threads has spawned successfuly.

Stage 3. Starting of the test. Duration of the test 300 sec

Stage 4. Processing of results...

Shutdown clients. Send TERM signal to 12962.

Response Time (s)

Transaction

%

Delivery

3.53

New Order

41.24

Order Status

3.86

Payment

39.23

Stock Level

3.59

Average : goth % Total Rollbacks

11- 1-

----

\section{:}
3.059

0.659:

1.175

0.684: 1.228

0.644:

1.161

0.652: 1.147

1603

18742

1756

17827

1630

0

172

0

0

0

%

0.00

0.92

0.00

0.00

0.00

3396.95 new-order transactions per minute (NOTPM)

5.5 minute duration

0 total unknown errors

31 second(s) ramping up

最重要的结果是输出信息中末尾处的一行:

3396.95 new-order transactions per minute (NOTPM)

这里显示了系统每分钟可以处理的最大事务数,越大越好(new-order 并非一种事务类型
的专用术语,它只是表明测试是模拟用户在假想的电子商务网站下的新订单)。

通过修改某些参数可以定制不同的基准测试。

.c

-t

到数据库的连接数。修改该参数可以模拟不同程度的并发性,以测试系统的可扩展性。

启用零延迟 (zero-delay)模式,这意味着在不同查询之间没有时间延迟。这可以对

数据库施加更大的压力,但不符合真实情况。因为真实的用户在执行一个新查询前

总需要一个“思考时间 (think time)”。

基准测试的持续时间。这个参数应该精心设置,否则可能导致测试的结果是无意义的。

•对于1/0密集型的基准测试,太短的持续时间会导致错误的结果,因为系统可能还

没有足够的时间对缓存进行预热。而对于CPU密集型的基准测试,这个时间又不应

该设置得太长;否则生成的数据量过大,可能转变成1/0密集型。

这种基准测试的结果,可以比单纯的性能测试提供更多的信息。例如,如果发现测试有
很多的回滚现象,那么就可以判定很可能什么地方出现错误了。

\subsection{Percona 的TPCC-MySQL 测试工具}
尽管sysbench 的测试很简单,并且结果也具有可比性,但毕竟无法模拟真实的业务压
力。相比而言,TPC-C测试则能模拟真实压力。2.5.4节谈到的dbt2是TPC-C的一个很
好的实现,但也还有一些不足之处。为了满足很多大型基准测试的需求,本书的作者重
新开发了一款新的类TPC-C测试工具,代码放在Launchpad上,可以通过如下地址获
取:https://code.launchpad.net/-percona-dev/perconatools/tpcc-mysgl,其中包含了一个
READMIE 文件说明了如何编译。该工具使用很简单,但测试数据中的仓库数量很多,可
能需要用到其中的并行数据加载工具来加快准备测试数据集的速度,否则这一步会花费
很长时间。

使用这个测试工具,需要创建数据库和表结构、加载数据、执行测试三个步骤。数据库
和表结构通过包含在源码中的SQL 脚本创建。加载数据通过用C写的tpcc\_

load工具完成,

该工具需要自行编译。加载数据需要执行一段时间,并且会产生大量的输出信息(一般
都应该将程序输出重定向到文件中,这里尤其应该如此,否则可能丢失滚动的历史信息)。
下面的例子显示了配置过程,创建了一个小型(五个仓库)的测试数据集,数据库名为
tpCC5。

$./tpcc\_load localhost tpcc5 username p4ssword 5

********************************米水水米*

***###easy###TPC-C Data Loader

***

*********************************米水米*

<Parameters>

Lserver」:localhost

[port]:3306

[DBname]:tpcc5

[user]:username

Lpass]:P4sSword

[warehouse]:5

TPCC Data Load Started...

Loading Item

•••

••• 5000

••••••••• 10000

15000

[output snipped for brevity]

Loading Orders for D=10,W= 5

..1000

2000

•.3000

Orders Done.

••.DATA LOADING COMPLETED SUCCESSFULLY.

,使用 tpcc\_start 工具开始执行基准测试。其同样会产
向到文件中。下面是一个简单的示例,使用五个线程操
测试时间:

$./tpcc.

-start localhost tpcc5 username p4ssword 5 5 3030

*************************************米米

*** ###easy### TPC-C Load Generator ***

***************************水***水**米水米水水

<Parameters>

[server]:1ocalhost

[port]:3306

[DBname]:tpcc5

Luserj:username

[passj:p4ssword

[warehouse」:5

[connection]:5

LFaMpUP: 30 (Sec.

measure: 30(sec.

RAMP-UP TIME.(30 sec.)

MEASURING START.

10,63(0):0.40,63(0):0.42,7(0):0.76,6(0):2.60, 6(0):0

20,75(0):0.40,74(0):0.62,7(0):0.04,9(0):2.38,7(0):0

30, 83(0):0.22, 84(0):0.37,9(0):0.04,7(0):1.97, 9(0):0

STOPPING THREADS.....

<RT Histogram

1.New-Order

2.Payment

3.Order-Status

4.De\_lvery

5.Stock-Level

<9oth Percentile RT (MaxRT)>

New-Order:0.37

(1.10)

Payment:0.47

(1.24)

Order-Status :0.06

(0.96)

Delivery :2.43

(2.72)

Stock-Level :0.75

(0.79)

<Raw Results>

[0] sC:221 1t:0 rt:0

f1:0

[1] sc:221 1t:0

rt:0

f1:0

[2] sC:23 1t:0 rt:0

f1:0

[3] sc:22 lt:0 rt:0f1:0

[4] sC:22 1t:0 rt:0f1:0

in 30 sec.

<Raw Results2(sum ver.)>

To] sc:221 1t:0 rt:0

f1:0

[1] sC:221 1t:0 rt:0f1:0

[2]

SC:23 1t:0 rt:0f1:0

[3] sC:22 1t:0 rt:0 f1:0

[4]sc:22 lt:0 rt:0f1:0

<Constraint Check> (all must be [OK])

[transaction percentage]

Payment:43.42%(>=43.0%)[OK]

Order-Status: 4.52%(>= 4.0%)[OK]

Delivery:4.32%(>= 4.0%)[OK]

Stock-Level:4.32%(>= 4.0%)[OK]

[response time (at least 90% passed)]

New-Order: 100.00%

[OK]

Payment:100.00%

[OK]

Order-Status:100.00%

[OK]

Delivery:100.00%[OK]

Stock-Level:100.00% [OK]

<TpmC>

442.000 TpmC

最后一行就是测试的结果:每分钟执行完的事务数进!!

。如果紧挨着最后一行前发现有异

常结果输出,比如有关于约束检查的信息,那么可以检查一下响应时间的直方图,或者
通过其他详细输出信息寻找线索。当然,最好是能使用本章前面提到的一些脚本,这样
就可以很容易获得测试执行期间的详细的诊断数据和性能数据。

\section{总结}
每个 MySQL 的使用者都应该了解一些基准测试的知识。基准测试不仅仅是用来解决业
务问题的一种实践行动,也是一种很好的学习方法。学习如何将问题分解成可以通过基
准测试来获得答案的方法,就和在数学课上从文字题目中推导出方程式一样。首先正确
地描述问题,之后选择合适的基准测试来回答问题,设置基准测试的持续时间和参数,
运行测试,收集数据,分析结果数据,这一系列的训练可以帮助你成为更好的MySQL
用户。

如果你还没有做过基准测试,那么建议至少要熟悉sysbench。可以先学习如何使用 oltp
和 fileio 测试。oltp 基准测试可以很方便地比较不同系统的性能。另一方面,文件系统
和磁盘基准测试,则可以在系统出现问题时有效地诊断和隔离异常的组件。通过这样的
注11:我们是在笔记本电脑上运行这个基准测试的,这只是作为演示用的。真实服务器的速度肯定比这
快得多。

基准测试,我们多次发现了一些数据库管理员的说法存在问题,比如SAN存储真的出
现了一块坏盘,或者RAID 控制器的缓存策略的配置并不是像工具中显示的那样。通过
对单块磁盘进行基准测试,如果发现每秒可以执行14000次随机读,那要么是碰到了严
重的错误,要么是配置出现了问题进12。

如果经常执行基准测试,那么制定一些原则是很有必要的。选择一些合适的测试工具并
深人地学习。可以建立一个脚本库,用于配置基准测试,收集输出结果、系统性能和状
态信息,以及分析结果。使用一种熟练的绘图工具如gnuplot或者R(不用浪费时间使
用电子表格,它们既笨重,速度又慢)。尽量早和多地使用绘图的方式,来发现基准测
试和系统中的问题和错误。你的眼睛是比任何脚本和自动化工具都更有效的发现问题的
工具。

注12:一块机械磁盘每秒只能执行几百次的随机读操作,因为寻道操作是需要时间的。


